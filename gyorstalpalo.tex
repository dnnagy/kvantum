\documentclass[12pt]{article}
\setlength{\textwidth}{17cm}
\setlength{\textheight}{24cm}
\setlength{\topmargin}{-2cm}
\setlength{\footskip}{1cm}
\setlength{\evensidemargin}{0cm}
\setlength{\oddsidemargin}{0cm}
\setlength{\parindent}{0cm}


\usepackage{amsmath}
\usepackage[magyar]{babel}
\usepackage[T1]{fontenc}
\usepackage[utf8]{inputenc}
\usepackage{fixltx2e}
\usepackage{multirow}
\usepackage{hyperref}
\usepackage{amsfonts}
\usepackage{amsthm}
\
\theoremstyle{plain}
\usepackage{graphicx}

%\usepackage{gensymb}
\usepackage{float}

%linkek 
\newcommand{\ndurl}[2]{\href{#1}{\color{red}{\underline{#2}}}}

%% New commands
\newcommand{\ket}[1]{\left| #1 \right >}
\newcommand{\bra}[1]{\left < #1 \right |}
\newcommand{\bracket}[2]{\left < #1 \middle | #2 \right>}
\newcommand{\norm}[1]{\left{||} #1 \right{||}}
\newcommand{\commut}[2]{\left [ #1 , #2 \right]}

%% Pauli matrices
\newcommand{\sigx}{\sigma_x}
\newcommand{\sigy}{\sigma_y}
\newcommand{\sigz}{\sigma_z}

\newcommand{\paulix}{
    \left( \begin{array}{cc}
        0 & 1 \\
        1 & 0
    \end{array}
    \right)
}
\newcommand{\pauliy}{
    \left( \begin{array}{cc}
        0 & -i \\
        i & 0
    \end{array}
    \right)
}
\newcommand{\pauliz}{
    \left( \begin{array}{cc}
        1 & 0 \\
        0 & -1
    \end{array}
    \right)
}
\newtheorem*{theorem*}{Tétel}
\newtheorem*{def*}{Definíció}
\newtheorem*{pld*}{Példa}
\newtheorem*{megj}{Megjegyzés}

\begin{document}

\title{Kvantummechanika gyorstalpaló}
\maketitle

\section{Bevezetés}
A kvantummechanika formalizmusának az alapja a $\mathcal{H}$ komplex Hilbert tér. A kvantummechanikai állapotok a $\mathcal{H}$ Hilbert tér
elemei, vektorai.

\subsection{Dirac-jelölés}
A $\mathcal{H}$ Hilbert tér elemei a $\ket{\psi}$ vektorok. A $\mathcal{H}^{*}$ duális térben a vektorokat $\bra{\phi}$
-vel jelöljük.
\\ A vektorok skalárszorzására a következő igaz:
\begin{itemize}
    \item $\bracket{\phi}{\psi} \in \mathbb{C}$
    \item $\bracket{\phi}{\psi} = \bracket{\psi}{\phi}^{*}$
    \item $ \norm \psi^2 = \bracket{\psi}{\psi} \geq 0$
\end{itemize}
További azonosságok:
\begin{itemize}
    \item $(c\ket\psi)^{*} = c^* \bra\psi$
\end{itemize}
\\ 
Ortonormált bázisról beszélünk, ha $\bracket{e_i}{e_j} = \delta_{ij}$
\subsubsection{Reprezentáció}
Számolások során a Hilbert-tér általában az $\mathcal{L}^2$-tér.
$$ \psi: \mathbb{R} \rightarrow \mathbb{C} \in \mathcal{L}^2 \iff \norm \psi^2 = \int\limits_{-\infty}^{\infty} |\psi(x)|^2 dx < \infty $$
\subsection{Kvantumállapot és Born-féle értelmezés}
A kvantummechanikában minden állapothoz rendelhető egy $\ket\psi$ vektor.
Az ilyen állapotok 1-re normált állapotok kell hogy legyenek, vagyis
$\norm \psi = \bracket{\psi}{\psi} = 1$.
Abban az esetben ha adott egy $\ket k$ bázis, és $\ket \psi$ felírható ezen a bázison:
$$\ket\psi = \sum\limits_{k=0}^{n} c_k\ket k$$
ahol  $c_k = \bracket{k}{\psi}$ és ez alapján
$$\ket \psi = \sum\limits_{k=0}^{n} \bracket{k}{\psi}\ket k $$

Ekkor a normálásból következik, hogy 
$$\bracket{\psi}{\psi} = \sum\limits_{k=0}^{n} c_k^* \sum\limits_{j=0}^{n} c_j \bracket{k}{j} =
\sum\limits_{k=0}^{n} |c_k|^2 = 1$$
továbbá
$$\sum\limits_{k=0}^{n} \ket k \bra k = \hat{1}$$

Fent $|c_k|^2$ annak a valószínűsége, hogy a $\ket \psi$ állapotban levő rendszer éppen $\ket k$ állapotban található.
Ez a valószínűség nem feltétlenül 0, mert a $\ket \psi$ egy szuperponált állapot.
\\ Tehát ha egy rendszer a $\ket a$ állapotban van preparálva és arra vagyok kíváncsi, hogy mi a valószínűsége annak,
hogy $\ket b$ állapotban találom, akkor a $p = |\bracket{b}{a}|^2$ értéket kell kiszámolni.
\subsection{Fiziaki mennyiségek és operátorok}
A $\mathcal{H}$ Hilbert téren értelmezhetők az $\hat{A} : \mathcal{H} \rightarrow \mathcal{H}$  típusú operátorok.
Értelmezhető az $\hat{A}$ operátor hermitikus adjungáltja a követkző képpen :
$$\bra \phi \hat{A} \ket \psi = \bra \psi \hat{A}^\dag \ket \phi$$
Itt $\hat{A}^\dag$ -ot az $\hat{A}$ hermitikus adjungáltjának nevezzük és igaz, hogy ha
$$\ket w = \hat{A}\ket v \implies \bra w = \bra v \hat{A}^\dag$$

\begin{def*}
    A $\hat{H}$ operátort hermitikusnak nevezzük ha $\hat H^\dag = \hat H$.
\end{def*}

\begin{theorem*}
    A kvantummechanikában minden fizikai állapothoz rendelhető egy hermitikus operátor.
\end{theorem*}

\begin{pld*}
    Az impulzus operátora $\hat p = -i\hbar \nabla$, az energia operátora $\hat H = \frac{\hat p^2}{2m} + \hat V$.
\end{pld*}

\begin{def*}
    Egy $\hat U$ operátort unitérnek nevezünk, ha $\hat U^{-1} = \hat U ^\dag$.
\end{def*}

\begin{pld*}
    Az időfejlődés operátora $\hat U(t) = e^{-\frac{i}{\hbar}\hat H t}$ unitér operátor.
\end{pld*}

A hermitikus operátorok sajátértékei mindig valósak. Ha egy operátor sajátvektorai ortonormált bázist alkotnak, akkor:
$$ \hat A = \sum\limits_{n} a_n \ket n \bra n \textrm{, ahol } \hat A \ket n = a_n \ket n $$
\begin{theorem*}
    Egy hermitikus operátor sajátvektorai mindig ortgonálisak.
\end{theorem*}
Illetve, ha egy $\hat A$ operátor sajátvektorai ortonormált bázist alkotnak, akkor egy tetszőleges $\ket \psi$ vektor
kifejthető ezen a bázison:
$$\hat A \ket n = a_n \ket n \textrm{, és} \bracket{n}{m} = \delta_{mn} \implies \ket\psi = \sum\limits_n \bracket{n}{\psi} \ket n 
\equiv  \sum\limits_n c_n \ket n$$ 
Legyen most $\hat A$ hermitikus, $\ket\psi$ egy állapot, úgy, hogy $\hat A \ket n = a_n\ket n$ és $\ket\psi = \sum\limits_n c_n \ket n$.
Ekkor az $\hat A \ket\psi$ vektor $b_m$ együtthatói az $\ket n$ bázison:
$$b_m = \bracket{m}{\hat A \psi} = \sum\limits_n \bra{m} \hat A \ket n \bracket{n}{\psi} = \sum\limits_n A_{mn}c_n$$
Ezért $$ A_{mn} = \bra m \hat A \ket n$$
$|\bra \phi \hat A \ket \psi|^2$ az minek a valószínűsége?

\subsection{Kommutátorok}
\begin{theorem*}
    Az $\hat A$ és $\hat B$ kommutátora $\commut{\hat A}{\hat B} = \hat A \hat B - \hat B \hat A$. 
\end{theorem*}
\begin{itemize}
    \item $\commut{\hat x_i}{\hat x_j} = 0$
    \item $\commut{\hat p_i}{\hat p_j} = 0$
    \item $\commut{\hat x_i}{\hat p_j} = i\hbar \delta_{ij}$
    \item $\commut{\hat L_i}{\hat L_j} = i \hbar \varepsilon_{ijk}\hat L_k$
    \item $\commut{\hat S_i}{\hat S_j} = i \hbar \varepsilon_{ijk}\hat S_k$
    \item $\commut{\hat J_i}{\hat J_j} = i \hbar \varepsilon_{ijk}\hat J_k$
    \item $\commut{\hat J^2}{\hat J_i} = \commut{\hat S^2}{\hat S_i} = \commut{\hat L^2}{\hat L_i} = 0$
    \item $\commut{\hat L_i}{\hat S_j} = 0$
    \item $\commut{\hat x_i}{\hat L_j} = i \hbar \varepsilon_{ijk}\hat x_k$
    \item $\commut{\hat p_i}{\hat L_j} = i \hbar \varepsilon_{ijk}\hat p_k$    
    \item Ha $\hat A$ hermitikus $\implies \hat A^\dag = \hat A \implies \commut{\hat A^\dag}{\hat A} = 0$  
    
    
\end{itemize}
\section{Harmonikus oszcillátor}
A harmonikus oszcillátor esetén a Hamilton-operátor ugyanz, mint a klasszikus Hamilton-függvény:
\[
  \hat H = \frac{\hat p^2}{2m} + \frac{1}{2}m\omega^2 \hat x^2  
\]
3D esetben:
\[
    \hat H = \frac{1}{2m}(\hat p_x^2+\hat p_y^2+\hat p_z^2) + \frac{1}{2}m\omega^2 (\hat x^2 + \hat y^2 + \hat z^2)  
\]
Az energiasajátértékekhez meg kell oldani a $\hat H \ket n = E_n \ket n$ egyenletet. Ennek a megoldása 1D-ben
folytonos bázison:
\[
    \ket n = \psi(x)_n = \frac{1}{\sqrt{(2^n n!)}} \cdot \left (\frac{m\omega}{\pi\hbar}\right)^{1/4}
    \cdot e^{-\frac{m\omega x^2}{2\hbar}} \cdot H_n\left(\sqrt\frac{m\omega}{\hbar}x\right)    
\]
ahol $H_n$-ek a Hermite-polinomok:
\[
    H_n(z) = (-1)^n e^z^2 \frac{d^n}{dz^n}e^{-z^2}
\]
Ezekből 
\[
    E_n = \left(n + \frac{1}{2}\right)   \hbar \omega 
\]
Az állapotok léptető operátorait a következő képpen definiáljuk:
\begin{itemize}
    \item $
        \hat a = \sqrt{\frac{m\omega}{2\hbar}}\left(\hat x + \frac{i}{m\omega}\hat p\right)
    $
    \item $
       \hat a^{\dag} = \sqrt{\frac{m\omega}{2\hbar}}\left(\hat x - \frac{i}{m\omega}\hat p\right)
    $
    \item $
        \commut{\hat a}{\hat a^{\dag}} = \hat 1
    $
    \item $
        \hat a \ket n = \sqrt n \ket{n-1}
    $
    \item $
        \hat a^{\dag} \ket n = \sqrt{n+1}\ket{n+1}
    $
    \item $
        \ket n = \frac{(\hat a^{\dag})^n}{\sqrt{n!}}\ket 0 
        \textrm{  folytonos bázison  } \ket 0 = Ce^{-\frac{m\omega x^2}{2\hbar}}
    $
\end{itemize}
\section{Impulzusmomentumok}
\subsection{Pálya-impulzusmomentum}
A kvantummechanikai impulzusmomentum operátora $\hat{\vec L} = \hat{\vec r} \times \hat{\vec p} = -i \hbar (\vec r \times \nabla)$.

Gömbi koordinátarendszerben felírva:
\begin{itemize}
    \item $
        \hat L^2 = -\hbar^2 \left(\frac{1}{\sin \theta} \frac{\partial}{\partial \theta} \left(\sin\theta\frac{\partial}{\partial \theta}\right) 
        + \frac{1}{\sin^2 \theta}\frac{\partial^2}{\partial\varphi^2}\right)
    $
    \item $
        \hat L_z = -i\hbar \frac{\partial}{\partial\varphi}
    $
    \item $
      \hat L_{\pm} = \hat L_x \pm i \hat L_y
    $
    \item $
        \hat L_+ = \hbar e^{i\varphi} \left( \frac{\partial}{\partial \theta} - i \cot\theta \frac{\partial}{\partial \varphi}\right)
    $
    \item $
        \hat L_- = \hbar e^{-i\varphi} \left( - \frac{\partial}{\partial \theta} - i \cot\theta \frac{\partial}{\partial \varphi}\right)
    $
    \item $
        \commut{\hat L_+}{\hat L_-} = 2 \hbar L_z
    $
    \item $
        \hat L^2 = \hat L_+ \hat L_- + \hat L_z^2 - \hbar \hat L_z
    $
\end{itemize}
Az $\hat L$ operátor sajátállapotait két kvantumszám határozza meg , $l$ és $m_l$, a sajátvektorokat ezért $\ket{l,m_l}$ jelöli.
A sajátértékek a következők lehetnek: $l\in \{0,1,2, ...\}$, $m_l \in \{-l,... 0, ..., l\}$.

Ezzel a jelöléssel igazak a következő azonosságok:
\begin{itemize}
    \item $
        \hat L^2 \ket{l, m_l} = \hbar^2 l (l+1) \ket{l,m_l}
    $
    \item $
        \hat L_z \ket{l, m_l} = \hbar m_l \ket{l,m_l}
    $
    \item $
      \hat L_{\pm}\ket{l, m_l} = \hbar \sqrt{l(l+1) - m_l(m_l\pm1)}\ket{l,m_l\pm 1}
    $
    \item $
        \hat L_+ \ket{l,m_l=l} = 0 ~~ \textrm{és} ~~ \hat L_- \ket{l,m_l=-l} = 0
    $
\end{itemize}
Ha $\ket{l,m_l}$-eket az $\mathcal{L}^2$ téren akarjuk ábrázolni, akkor:
\[
    \bracket{\theta, \varphi}{l,m_l} = Y^{m_l}_l(\theta, \varphi)
\]
\[
   \bracket{l',m'_l}{l,m_l} = \int \limits_0^{\pi}\sin\theta d \theta \int \limits_0^{2\pi}d \varphi (Y^{m_l'}_{l'})^{*} Y^{m_l}_l = \delta_{ll'}\delta_{m_lm_l'}
\]
Az $Y^{m_l}_l(\theta, \varphi)$ függvények a \ndurl{https://en.wikipedia.org/wiki/Spherical_harmonics}{gömbi harmonikusok}
\subsection{Spin}
A spin-operátort $\hat S$ jelöli, sajátállapotait két kvantumszám adja meg, $s$ és $m_s$, úgy, \\ hogy
$s\in\{0, \frac{1}{2}, 1, \frac{3}{2}, ...\}$ és $m_s \in \{-s, ... 0, ..., s\}$
\subsubsection{Az $s = \frac{1}{2}$ spin és a Pauli-mátrixok}
$\paulix\pauliy\pauliz$
\subsection{Teljes impulzusmomentum}
\subsection{Impulzusmomentumok összeadása és Clebsch–Gordan együtthatók}
Az pálya-impulzusmomentum és a spin különböző Hilbert-tér elemei, ezért ha össze akarom adni őket,
akkor a Hilbert-terek direkt szorzatát kell képezni.
Legyen egy rendszer (pl. H-atom elektronja) ami spinnel és pálya-impulzusmomentummal is rendelkezik.
Legyenek a spin sajátállapotok $\ket{s, m_s}$ illetve a pálya-impulzusmomentum sajátállapotok $\ket{l, m_l}$.
Ekkor a teljes impulzusmomentum sajátállapotai a két tér elemeinek direkt szorzata:
\[
    \ket{j, m} = \ket{l, m_l} \otimes \ket{s, m_s} \equiv \ket{l, m_l}\ket{s, m_s} \equiv \ket{l, m_l,s, m_s}
\] 

Ezen a szorzattéren hat a $\hat{\vec J}$ operátor, amit így definiálunk:
\[
    \hat{\vec J} = \hat{\vec L} + \hat{\vec S}
\]
Ez a jelölés pongyola, mert a $\hat{\vec J}$-nek a szorzattéren vagyis $\mathcal{H}_L \otimes \mathcal{H}_S$-en kell hatnia,
ezért korrektül:
\[
    \hat J_k = \hat L_k \otimes \hat 1 + \hat 1 \otimes \hat S_k
\]
Ez a következő képpen hat a sajátállapotokra:
\[
    \hat J^2 \ket{j, m} = \hbar^2 j(j+1)\ket{j,m}
\]
\[
    \hat J_z \ket{j, m} = \hbar m\ket{j,m}
\]
Szintén be lehet vezetni a léptető operátorokat:
\begin{itemize}
    \item $
        \hat J_{\pm} = \hat J_x \pm  i \hat J_y
    $
    \item $
       \hat J_+ \hat J_-  = \hat J ^2 - \hat J_z^2 + \hbar \hat J_z
    $
    \item $
       \hat J_- \hat J_+  = \hat J ^2 - \hat J_z^2 - \hbar \hat J_z
    $
    \item $
        \commut{\hat J_+}{\hat J_-} = 2 \hbar \hat J_z
    $
    \item $
        \commut{\hat J_z}{\hat J_{\pm}} = \pm \hbar \hat J_{\pm}
    $
    \item $
        \hat J_{\pm}\ket{j,m} = \hbar \sqrt{j(j+1) - m(m\pm1)}\ket{j,m\pm1}
    $
\end{itemize}
\section{Hidrogén-atom}
\begin{megj}
    Ha a főkvantumszám $n$, akkor $l_{max}=n-1$ vagyis $l \in \{0,1,...,n-1\}$
\end{megj}
\section{Perturbációszámítás}
\subsection{Időfüggetlen perturbációszámítás}
Adott egy $\hat H$ operátor ami úgy néz ki, hogy $\hat H = \hat H^{(0)} + \lambda \hat H'$, ahol $\lambda$ kicsi, 
$\lambda\hat H'$ a perturbáció, $\hat H^{(0)}$-nak pedig ismerjük a sajátállapotait:
\[
    \hat H^{(0)} \ket{n^{(0)}} = E_n^{(0)} \ket{n^{(0)}} ~, ~~ n=1,2,... 
\]
A $\hat H$ sajátérték-problémájának megoldása így fog kinézni sorfejtett alakban:
\[
    E_n = \lambda^0 E_n^{(0)} + \lambda^1 E_n^{(1)} + \lambda^2 E_n^{(2)} + ...
\]
\[
  \ket n = \lambda^0 \ket{n^{(0)}} + \lambda^1 \ket{n^{(1)}} + \lambda^2 \ket{n^{(2)}} + ...  
\]
Az elsőrendű korrekciók a fentiek alapján:
\[
    E_n^{(1)} = \bra{n^{(0)}} \hat H' \ket{n^{(0)}}      
\]
\[
    \ket{n^{(1)}} = \sum\limits_{k\neq n}\frac{\bra{k^{(0)}} \hat H' \ket{n^{(0)}}}{E_n^{(0)}-E_k^{(0)}}
    \ket{k^{(0)}}
\]

Másodrendű energia-korrekció:
\[
    E_n^{(2)} = \sum\limits_{k\neq n}\frac{\left|\bra{k^{(0)}} \hat H' \ket{n^{(0)}}\right|^2}{E_n^{(0)}-E_k^{(0)}}    
\]

\textbf{Figyelem!} A fenti korrekciók csak nemdegenerált energiaspektrum esetében igazak.
\end{document}
